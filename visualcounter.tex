\setuplayout[
                   width=middle,
                  height=middle,
                %location=middle,
                topspace=0.5in,
             bottomspace=.75in,
          bottomdistance=.25in,
                  bottom=.25in,
               backspace=1.0in,
                cutspace=1.0in,
              leftmargin=0.55in,
             rightmargin=0.55in,
      leftmargindistance=0.1in,
     rightmargindistance=0.1in,
                  header=0.25in,
                  footer=0.5in,
           headerdistace=0.25in,
          footerdistance=0.25in,
                 marking=on,
%                   grid=yes,
    ]

\setuppagenumbering [location=footer]


\setupbodyfontenvironment[default][em=italic]
\setupbodyfont[dejavu,10pt]


\definetype[command][color=darkred, style=type]
\definetype[options][color=darkblue, style=type]

\setuptyping[option=tex, before=\startEXAMPLE, after=\stopEXAMPLE]

\setupindenting[medium,yes]
\setupwhitespace[medium]

\setuphead[title][alternative=middle, textstyle=\sansbold] 
\setuphead[section,subject,subsubject,subsection]
          [numberstylea=\sansbold,textstyle=\sansbold]

\setupitemize[1][autointro]
\setupitemize[indenting=no]

%% Frames and Backgrounds : <<<

\definetextbackground
    [EXAMPLE]
    [           mp=background:random,
          location=paragraph,
     rulethickness=1pt,
        framecolor=darkred,
             width=broad,
            height=fit,
        leftoffset=5em,
       rightoffset=5em,
            before={\testpage[3]\blank[2*big]},
             after={\blank},
    ]

\startuseMPgraphic{background:random}
   path p;
   for i = 1 upto nofmultipars :
    p = (multipars[i]
     topenlarged 8pt
     bottomenlarged 4pt
     leftenlarged -4EmWidth
     rightenlarged -4EmWidth ) randomized 4pt ;
   fill p withcolor lightgray ;
   draw p withcolor \MPvar{linecolor}
    withpen pencircle scaled \MPvar{linewidth};
   endfor;
   setbounds currentpicture to boundingbox OverlayBox ;
\stopuseMPgraphic

\defineframedtext
    [EXAMPLEframe]
    [rulethickness=1pt,
        framecolor=darkred,
            height=6.55cm,
             width=broad,
        background=color,
   backgroundcolor=gray,
    ]


\usemodule[visualcounter]

\startbuffer[visual]
\definevisualcounter
  [visualitem]
  [scratchmarks]
  [
    counter=\currentitemgroupcounter,
    width=1.5bp,
    height=1.2ExHeight,
    distance=3bp,
  ]

\definesymbol[visual][\usevisualcounter{visualitem}]
\stopbuffer

\getbuffer[visual]

\starttext


\title {The \command{visualcounter} module \\ Aditya Mahajan}

\starttyping
\usemodule[visualcounter]
\stoptyping

\startitemize[visual, broad, packed, columns, two]
  \item Find \TEX\ documents to be too boring?
  \item Voilà, the \command{visualcounter} module!
  \item Make your presentations stand out.
  \item Turn any counter into a picture.
\stopitemize

The above effect was achieved by first defining a \command{visualitem} counter
and a symbol \command{visual} that uses that counter:

\typebuffer[visual]

and then using the symbol \command{visual} in an itemization:

\starttyping
\startitemize[visual, ...]
    \item ...
    \item ...
    ...
\stopitemize
\stoptyping

\subject {The \command{scratchmarks} counter}

\subject {The \command{markers} counter}

\subject {The \command{pulseline} counter}

\subject {The \command{mayanumbers} counter}

\stoptext
