\usemodule[documentation]
\usemodule[visualcounter]

\startbuffer[visual]
\definevisualcounter
  [visualitem]
  [scratchmarks]
  [
    counter=\currentitemgroupcounter,
    width=1.5bp,
    height=1.2ExHeight,
    distance=3bp,
  ]

\definesymbol[visual][\usevisualcounter{visualitem}]
\stopbuffer

\startbuffer[page]
\definevisualcounter
  [visualpage]
  [mayanumbers]
  [
    counter=userpage,
    maxwidth=\textwidth,
  ]

\stopbuffer

\startbuffer[pagenumber]
\setupfootertexts[\usevisualcounter{visualpage}]
\stopbuffer


\getbuffer[visual,page,pagenumber]

\starttext


\title {The \command{visualcounter} module \\ Aditya Mahajan}

\starttyping
\usemodule[visualcounter]
\stoptyping

\startitemize[visual, broad, packed, columns, two]
  \item Find \TEX\ documents to be too boring?
  \item Voilà, the \command{visualcounter} module!
  \item Make your presentations stand out.
  \item Turn any counter into a picture.
\stopitemize

The above effect was achieved by first defining a \command{visualitem} counter
and a symbol \command{visual} that uses that counter:

\typebuffer[visual]

and then using the symbol \command{visual} in an itemization:

\starttyping
\startitemize[visual, ...]
    \item ...
    \item ...
    ...
\stopitemize
\stoptyping

Notice the counter used for page numbering? That was achieved by first defining
a \command{visualpage} counter:

\typebuffer[page]

and setting it as the footer text:

\typebuffer[pagenumber]

The above examples show the basic usage of the modules. The module provides two
commands: \type{\definevisualcounter} to define a visual counter

\starttyping
\definevisualcounter
    [...]     % [[\comment{name of the counter}]]
    [...]     % [[\comment{optional name of the parent counter} ]]
    [
     ...=..., % [[\comment{key-value settings} ]]
    ]
\stoptyping

and \type{\usevisualcounter} to use an already defined counter

\starttyping
\usevisualcounter{...}  % [[\comment{name of the counter}]]
\stoptyping

\subject {The \command{scratchmarks} counter}

\subject {The \command{markers} counter}

\subject {The \command{pulseline} counter}

\subject {The \command{mayanumbers} counter}

\stoptext
