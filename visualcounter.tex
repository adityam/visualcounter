\usemodule[documentation]
\usemodule[visualcounter]

\definevisualcounter
  [visualsection]
  [countdown]
  [n={\somenamedheadnumber{visualsection}{current}},
   last={\somenamedheadnumber{visualsection}{last}},
   maxtext={99},
   text={\headnumber[visualsection]},
   style=\ssbf,
  ]
  
\definehead[visualsection]
           [subsection]
           [numbercommand=\visualsectionnumber,
            numnberstyle=sansbold,
            textstyle=sansbold,
            page=yes]

\def\visualsectionnumber#1{\smash{\framed[location=middle,frame=off]{\usevisualcounter[text=#1]{visualsection}}}}

\startbuffer[visual]
\definevisualcounter
  [visualitem]
  [scratchmarks]
  [ counter=\getvalue{v_strc_itemgroups_counter},
    width=1.5bp,
    height=1.2ExHeight,
    distance=3bp]

\definesymbol[visual][\usevisualcounter{visualitem}]
\stopbuffer

\startbuffer[page]
\definevisualcounter
  [visualpage]
  [mayanumbers]
  [
    counter=userpage,
    maxwidth=\textwidth,
  ]

\stopbuffer

\startbuffer[pagenumber]
\setupfootertexts[\usevisualcounter{visualpage}]
\stopbuffer


\getbuffer[visual,page,pagenumber]

\startbuffer[brightred]
\definecolor[bright-red] [h=DE1B1B]
\definecolor[dull-black] [h=2B2B2B]
\definecolor[dull-yellow][h=E9E581]

\definepalet
   [brightred]
   [past=dull-black, active=bright-red, future=dull-yellow]
\stopbuffer

\def\showcounter[#1]%
    {\startmiddlealigned
      \startcombination[#1]
        {\usevisualcounter[n=3, last=12]{dummy}}{3 out of 12}
        {\usevisualcounter[n=4, last=12]{dummy}}{4 out of 12}
        {\usevisualcounter[n=5, last=12]{dummy}}{5 out of 12}
        {\usevisualcounter[n=6, last=12]{dummy}}{6 out of 12}
      \stopcombination
    \stopmiddlealigned}

\starttext


\title {The \command{visualcounter} module \\ Aditya Mahajan \\ \currentdate}

\starttyping
\usemodule[visualcounter]
\stoptyping

\startitemize[visual, broad, packed, columns, two]
  \item Find \TEX\ documents to be too boring?
  \item Voilà, the \command{visualcounter} module!
  \item Make your presentations stand out.
  \item Turn any counter into a picture.
\stopitemize

\page

The above effect was achieved by first defining a \command{visualitem} counter
and a symbol \command{visual} that uses that counter:

\typebuffer[visual]

and then using the symbol \command{visual} in an itemization:

\starttyping
\startitemize[visual, ...]
    \item ...
    \item ...
\stopitemize
\stoptyping

\page

Notice the counter used for page numbering? That was achieved by first defining
a \command{visualpage} counter:

\typebuffer[page]

and setting it as the footer text:

\typebuffer[pagenumber]

\page

The above examples show the basic usage of the modules. The module provides two
commands: \type{\definevisualcounter} to define a visual counter

\starttyping
\definevisualcounter
    [...]     % [[\comment{name of the counter}]]
    [...]     % [[\comment{optional name of the parent counter} ]]
    [
     ...=..., % [[\comment{key-value settings} ]]
    ]
\stoptyping

and \type{\usevisualcounter} to use an already defined counter

\starttyping
\usevisualcounter
  [...=...] % [[\comment{key-value settings}]]
  {...}     % [[\comment{name of the counter}]]
\stoptyping

\section {So, how do I use this?}

Visual counters are defined and used in two ways:
\startitemize[visual, broad, joinedup]
  \item Using a low level interface that explicitly sets the current values of
    the counter, last count, the \METAPOST\ graphic that draws the counter, and
    the color palette. 
  \item Using higher-level interfaces built on top of the low-level interface
    that allows you to specify a \emph{structure counter} like those used for
    page numbering, itemizations, descriptions, etc.
\stopitemize

\page

\subsection {The low-level interface}

To begin with, lets not worry about how to define \METAPOST\ graphics that draw
the counter. The module provides a predefined set of \command{visualcounter}s,
and, for now, we'll just use those: the \command{scratchmarks}  counter. Details
on defining new counters are explained later.

\startbuffer[scratchmarks]
\usevisualcounter[n=6, last=12]{scratchmarks}
\stopbuffer

Suppose that I want to show that I am on page 6 out of 12 pages:

\startmiddlealigned
  \getbuffer[scratchmarks]
\stopmiddlealigned

which uses a predefined counter \command{scratchmarks} and was typed as follows:

\typebuffer[scratchmarks]

The counter may be made smaller

\startbuffer[smaller]
\setupvisualcounter[scratchmarks]
                   [width=1pt, height=8pt, distance=2pt]
\stopbuffer

\startmiddlealigned
  \getbuffer[smaller,scratchmarks]
\stopmiddlealigned

or may use a different color palette

\startbuffer[palette]
\setupvisualcounter[scratchmarks][palette=brightred]
\stopbuffer

\startmiddlealigned
  \getbuffer[brightred,palette,scratchmarks]
\stopmiddlealigned

\page

These settings are changed using \command{\setupvisualcounter}. In particular, to
get a small counter, use: 

\typebuffer[smaller]

and to change the color palette, use:

\typebuffer[palette]

where the \type{brightred} palette was defined as

\typebuffer[brightred]

\subsection {The high-level interface}

The high level interface is useful to display a \CONTEXT{} counter. Rather than
manually setting the value of \options{n}, \options{last}, \options{text}, and
\options{maxtext} (the last two are used only by a few counters), simply set the
value of a \options{counter} and the other values are automatically generated.
As an example, recall the set up for displaying the page numbers in the footer:

\typebuffer[page]

In the above example, \command{userpage} is the name of the counter that keeps
track of the user page number.

\page

A list of some commonly used \CONTEXT{} counters is given below:

\startbuffer[internal] counter=\getvalue{v_strc_itemgroups_counter},
\stopbuffer


\starttabulate[|l|l|]
  \NC Page numbers \NC \type{userpage} \NC \NR
  \NC Item group numbers 
      \NC \type{\v_strc_itemgroups_counter}\footnote[internal]
          {Macros names with underscore are internal \CONTEXT{} macros, and generally are
          not meant to be used in user code. The easiest way to set the value of
          \options{counter} to \options{\v_strc_itemgroups_counter} is to use:
          \typebuffer[internal]} \NC \NR
  \NC Enumeration number\NC name of the enumeration \NC \NR
\stoptabulate


\section {Changing the color of counters}

All counters use three colors: \command{past} to display past counters,
\command{active} to display current counter, and \command{future} to display
future counters. To change the color scheme, first define a new color palette
(which is misspelled in \CONTEXT{} is \command{colorpalet}):

\starttyping
  \definecolorpalet
      [...], % [[\comment{Name of the color palette}]]
      [
        past=..., % [[\comment{any previously defined \CONTEXT{} color}]]
      active=..., % [[\comment{any previously defined \CONTEXT{} color}]]
      future=..., % [[\comment{any previously defined \CONTEXT{} color}]]
      ]
\stoptyping

\page

and then set the color palette of a particular counter either using

\starttyping
  \definevisualcounter
      [...]     % [[\comment{name of the counter}]]
      [...]     % [[\comment{optional name of the parent counter} ]]
      [
        ...,
        palette=..., % [[\comment{name of a previously defined} \options{colorpalet}]]
        ...
      ]
\stoptyping

or using

\starttyping
  \setupvisualcounter
      [...]     % [[\comment{name of the counter}]]
      [
        ...,
        palette=..., % [[\comment{name of a previously defined} \options{colorpalet}]]
        ...
      ]
\stoptyping

\section {Changing the size of the counters}

All predefined counters have tunable parameters (such as \options{width},
\options{height}, and \options{distance}) that change the size of the counter.
The default sizes are given in terms of \options{EmWidth} or \options{ExHeight};
so they adapt to the size of the surrounding text. 

\section {Changing the style of text displayed in the counters}

Some visual counters (currently, only the \command{countdown} counter) also
display text. The style of this text may be set using:


\starttyping
  \setupvisualcounter
      [...]     % [[\comment{name of the counter}]]
      [
        ...,
        style=..., %[[ \comment{any valid \CONTEXT{} style}]]
        color=..., %[[ \comment{any valid \CONTEXT{} color}]]
        ...
      ]
\stoptyping

Note that changing the font size in the \options{style} affects the value of
\options{EmWidth} and \options{ExHeight} and therefore also scales the counter
appropriately. If this is not desirable, then you also need to set the size of
the counter in dimensions that are independent of bodyfontsize.

\section {Predefined counters}

The \command{visualcounter} module provides the following predefined counters:

\startnarrower
\placelist[visualsection][criterium=local, alternative=f]
\stopnarrower

\page

\visualsection {The \command{scratchmarks} counter}

The \command{scratchmarks} counter is inspired by the \command{fuzzycount}
counter that is part of the \CONTEXT's metapost library \command{txt}. The output
looks as follows:

\definevisualcounter
  [dummy]
  [scratchmarks]

\showcounter[4]

\blank[2*big]

The \command{scratchmarks} counter has the following tunable parameters:
\startitemize[packed, joinedup]
  \item \options{width} (default \options{3bp}): the width of each stroke
  \item \options{height} (default \options{3ExHeight}): 
    the length of the marker (and strictly speaking, not
    the height; the real height is $\options{height}×\sin(\options{angle})$).
  \item \options{distance} (default \options{0.5EmWidth}):
    the distance between two successive markers. The
    distance is measured from the middle of one marker to the middle of the
    other (that is, it does not take the width of the stroke into account).
  \item \options{angle} (default \options{75}): the angle of the forward
    markers. The angle of the backward marker is \command{-angle}.
    Only angles between \options{-90} and \options{90} give proper output. 
\stopitemize

\page

For example, the output with \options{width=1.5bp, angle=45} is:

{\setupvisualcounter[dummy][width=1.5bp, angle=45]\showcounter[4]}

\blank[2*big]

An angle less than \options{0} changes the direction of the stroke. 
For example the output with \options{width=1.5bp, angle=-45} is:

{\setupvisualcounter[dummy][width=1.5bp, angle=-45]\showcounter[4]}


\visualsection {The \command{mayanumbers} counter}

The \command{mayanumbers} counter is inspired by the Mayan numbering system that I saw in the
documentary \quotation{Breaking the Maya code}. It counter does not strictly
follow the Mayan numbering system. The Mayan numbering system is written
vertically; the output of this counter is horizontal which makes it more useful
for displaying page numbers in presentations.

\definevisualcounter
  [dummy]
  [mayanumbers]

\showcounter[2*2]

\blank[2*big]

The shape of the small and the large markers is as follows:

\startmiddlealigned
  \startcombination[2]
    {\usevisualcounter[n=1,last=5,width=3EmWidth, height=3ExHeight]{dummy}}
    {The shape of the small marker}
    {\usevisualcounter[n=5,last=5,width=3EmWidth, height=3ExHeight]{dummy}}
    {The shape of the large marker}
  \stopcombination
\stopmiddlealigned

\page

The \command{mayanumbers} counter has the following tunable parameters:
\startitemize[packed, joinedup]
  \item \options{width} (default value \options{1EmWidth}): The width of the
    small marker. 
  \item \options{height} (default value \options{1ExHeight}): The height of the
    markers.
  \item \options{distance} (default value \options{width/4}): The distance
    between two small markers. The distance between each group of four small
    counters is \command{2*distance}.
\stopitemize

\blank[2*big]

For example, to get an output that is half the default size, use 
the options \goodbreak \options{width=0.5EmWidth, height=0.5ExHeight}.

{\setupvisualcounter[dummy][width=0.5EmWidth, height=0.5ExHeight]\showcounter[4]}

% {\setupvisualcounter[dummy][style=\tfxx]\showcounter[4]}

\visualsection {The \command{countdown} counter}

This counter is inspired by the spinning wheel on the iPhone and on Google
images.

\definevisualcounter
  [dummy]
  [countdown]

\startmiddlealigned
  \startcombination[4]
    {\framed[frame=off,width=0.2\textwidth]{\usevisualcounter[n=3, last=12]{dummy}}}{3 out of 12}
    {\framed[frame=off,width=0.2\textwidth]{\usevisualcounter[n=4, last=12]{dummy}}}{4 out of 12}
    {\framed[frame=off,width=0.2\textwidth]{\usevisualcounter[n=5, last=12]{dummy}}}{5 out of 12}
    {\framed[frame=off,width=0.2\textwidth]{\usevisualcounter[n=6, last=12]{dummy}}}{6 out of 12}
  \stopcombination
\stopmiddlealigned

\blank[2*big]

The \command{countdown} counter may also be used with a text display of the
counter.

\startmiddlealigned
  \startcombination[4]
    {\framed[frame=off,width=0.2\textwidth]{\usevisualcounter[n=3, text=3, last=12]{dummy}}}{3 out of 12}
    {\framed[frame=off,width=0.2\textwidth]{\usevisualcounter[n=4, text=4, last=12]{dummy}}}{4 out of 12}
    {\framed[frame=off,width=0.2\textwidth]{\usevisualcounter[n=5, text=5, last=12]{dummy}}}{5 out of 12}
    {\framed[frame=off,width=0.2\textwidth]{\usevisualcounter[n=6, text=6, last=12]{dummy}}}{6 out of 12}
  \stopcombination
\stopmiddlealigned

\blank[2*big]

The example below shows how the counter changes when the number of steps are
increased.

\startmiddlealigned
  \startcombination[4]
    {\framed[frame=off,width=0.2\textwidth]{\usevisualcounter[n=13, last=24]{dummy}}}{13 out of 24}
    {\framed[frame=off,width=0.2\textwidth]{\usevisualcounter[n=14, last=36]{dummy}}}{14 out of 36}
    {\framed[frame=off,width=0.2\textwidth]{\usevisualcounter[n=25, last=48]{dummy}}}{25 out of 48}
    {\framed[frame=off,width=0.2\textwidth]{\usevisualcounter[n=36, last=60]{dummy}}}{36 out of 60}
  \stopcombination
\stopmiddlealigned

\page

The \command{countdown} counter has the following tunable parameters:
\startitemize[packed, joinedup]
  \item \options{width} (default value \options{1EmWidth}) and \options{height}
    (default value \options{1ExHeight}): The maximum of these two determine the
    difference between the inner and outer diameter of the ring.

  \item \options{text} (not set by default): The text to be displayed in the
    middle of the counter.

  \item \options{maxtext} (not set by default): The diameter of the inner circle
    is equal to \options{1.5} times the maximum of the width and the height of
    \options{maxtext}.

  \item \options{distance} (default value \options{3EmWidth}): 
    The distance between the consecutive markers along the circumference of the
    outer circle is equal to \options{distance/last}. 
\stopitemize

\blank[2*big]

For example, to get a continuous circle set \options{distance=0pt}:

\startmiddlealigned
  \setupvisualcounter[dummy][distance=0cm]
  \startcombination[4]
    {\framed[frame=off,width=0.2\textwidth]{\usevisualcounter[n=13, last=24]{dummy}}}{13 out of 24}
    {\framed[frame=off,width=0.2\textwidth]{\usevisualcounter[n=14, last=36]{dummy}}}{14 out of 36}
    {\framed[frame=off,width=0.2\textwidth]{\usevisualcounter[n=25, last=48]{dummy}}}{25 out of 48}
    {\framed[frame=off,width=0.2\textwidth]{\usevisualcounter[n=36, last=60]{dummy}}}{36 out of 60}
  \stopcombination
\stopmiddlealigned

When the high level interface is used, i.e., when the option
\options{counter=...} is set, the value of \options{text} is the converted value of the
counter, and the value of \options{maxtext} is the converted value of last
counter.

\visualsection {The \command{markers} counter}

The \command{markers} counter is inspired by the section markers displayed by
the \LATEX{} \options{beamer} package. This module comes in two shapes: the
default shape is circle

\definevisualcounter
  [dummy]
  [markers]

\showcounter[1*4]

\page

The alternative shape is a square, which is selected using

\starttyping
  \setupvisualcounter
      [...]     % [[\comment{name of the counter}]]
      [
        ...
        mpsetups=[[\options{visualcounter::markers:square}]],
        ...
      ]
\stoptyping

{\setupvisualcounter[dummy][mpsetups=visualcounter::markers:square]\showcounter[1*4]}

\page

The \options{visualcounter::markers:square} is a predefined alternative shape.
Before discussing how to define a new shape, lets look at the tunable parameters
of the \command{markers} counter:
\startitemize[packed, joinedup]
  \item \options{width} (default value \options{1EmWidth}): The width of the
    each marker.

  \item \options{distance} (default value \options{width/4}): The edge to edge
    distance between successive markers.

  \item \options{mpsetups} (default value \options{visualcounter::markers:circle}): 
    A \command{useMPgraphic} that determines the shape and the display of the
    markers. The module comes with three predefined shapes,
    \options{visualcounter::markers:circle}, \goodbreak
    \options{visualcounter::markers:square}, and
    \options{visualcounter::markers:star}.
\stopitemize

To define a new shape, you have to create a \command{useMPgraphic} that defines
three \METAPOST{} macros:

\startitemize[packed, joinedup]
  \item \options{show_past_marker(expr shift)}
  \item \options{show_active_marker(expr shift)}
  \item \options{show_future_marker(expr shift)}
\stopitemize

The argument to these macros is the shift calculated based on their position.
The macros are responsible for using the shift amount either for horizontal
shift  or for vertical shift, and coloring the markers with appropriate colors.
The three colors from the color palette are available as \options{past_color},
\options{active_color}, \options{future_color}.

\page

\startbuffer[star-definition]
\startuseMPgraphic{visualcounter::markers:star}
% [[\comment{Copied from http://tug.org/pracjourn/2012-1/hefferon.html}]]
def fullstar = 
  hide (
    z0  = origin; 
    z1  = (x0, 0.5); 
    z2  = ((z1 - z0) rotated (  360/5)) + z0 ;
    z3  = ((z1 - z0) rotated (2*360/5)) + z0 ;
    z4  = ((z1 - z0) rotated (3*360/5)) + z0 ; 
    z5  = ((z1 - z0) rotated (4*360/5)) + z0 ;
    z6  = whatever[z1, z3] = whatever[z2, z5];
    z7  = whatever[z1, z3] = whatever[z2, z4];
    z8  = whatever[z2, z4] = whatever[z3, z5];
    z9  = whatever[z1, z4] = whatever[z3, z5];
    z10 = whatever[z1, z4] = whatever[z2, z5];
  ) 
  (z1 -- z6 -- z2 -- z7 -- z3 -- z8 -- z4 
      -- z9 -- z5 -- z10 -- cycle)
enddef;

def show_star(expr shift, fill_color) = 
  newpath p;
  p := fullstar xyscaled(width, width) shifted (shift, 0);

  fill p withcolor fill_color;
  draw p withcolor 0.5*fill_color;
enddef;

def show_past_marker(expr shift) = 
    show_star(shift, past_color);
  enddef;

def show_active_marker(expr shift) = 
    show_star(shift, active_color);
enddef;

def show_future_marker(expr shift) = 
    show_star(shift, future_color);
enddef;
\stopuseMPgraphic
\stopbuffer

\startbuffer[star-counter]
\definevisualcounter
  [stars]
  [markers]
  [mpsetups=visualcounter::markers:star, % [[\comment{use "star" marker}]]
   width=1.5EmWidth,
   distance=0.25EmWidth,
   palette=star-colors, % [[\comment{defined elsewhere}]]
   last=5, % [[\comment{rating out of 5}]]
  ]
\stopbuffer

\definepalet
  [star-colors]
  [active=orange,
     past=orange,
   future=gray]

\getbuffer[star-definition, star-counter]

As an example, lets consider a new visual counter that displays a
\quotation{star rating}: 

\startmiddlealigned
  \startcombination[3]
    {\framed[frame=off,width=0.2\textwidth]{\usevisualcounter[n=1]{stars}}}{1 out of 5}
    {\framed[frame=off,width=0.2\textwidth]{\usevisualcounter[n=3]{stars}}}{3 out of 5}
    {\framed[frame=off,width=0.2\textwidth]{\usevisualcounter[n=5]{stars}}}{5 out of 5}
  \stopcombination
\stopmiddlealigned

For better visual effect, I have also changed the color palette to one that uses
same color for \options{past} and \options{current} markers. Note
that the above alternative is available as the
\options{visualcounter::markers:star} option to \options{mpsetups}. The following
description is just to explain how to define a new marker.

\page

To create such a counter, first create a \command{useMPgraphic} as follows (I
could have used any name instead of \options{visualcounter::markers:star}):

\adaptlayout[lines=+2]

\typebuffer[star-definition]

\page

Next, define a new \command{visaulcounter} that inherits from \command{markers}
but uses a different \options{mpsetups}:

\typebuffer[star-counter]

\startbuffer[star-example]
\usevisualcounter[n=4]{stars}
\stopbuffer

To use this counter, type:

\typebuffer[star-example]

which gives: \framed[frame=off,location=lohi]{\getbuffer[star-example]}



\visualsection {The \command{progressbar} counter}

The \command{progressbar} counter is based on a question on TeX.SE. This counter
is not yet finalized. I am still working on the interface to change the shape of
the progress bar, and perhaps add some shading.

\definevisualcounter
  [dummy]
  [progressbar]

\showcounter[1*4]


\visualsection {The \command{pulseline} counter}

The \command{pulseline} counter is based on the heart pulses shown in a heart
rate measurement device. This counter is not yet finalized.

\definevisualcounter
  [dummy]
  [pulseline]

\definepalet
  [green]
  [past=darkgreen,
   active=darkgreen,
   future=darkgreen]

{\setupvisualcounter[dummy][rulethickness=2bp,palette=green]\showcounter[2*2]}

% \section[sec:new-counters]{How to define new counters}
% 
% The name of the \command{useMPgraphic} must start with \options{visualcounter::}
% (note the double \options{::}), so that the graphic uses the
% \options{visualcounter} MP instance.


\stoptext
