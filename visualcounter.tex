\usemodule[documentation]
\usemodule[visualcounter]

\definevisualcounter
  [section]
  [countdown]
  [n={\somenamedheadnumber{section}{current}},
   last={\somenamedheadnumber{section}{last}},
   text={\headnumber[section]},
  ]
  
\setuphead[numbercommand=\visualsection, alternative=inmargin]

\def\visualsection#1{\smash{\framed[location=middle,frame=off]{\usevisualcounter[text=#1]{section}}}}

\startbuffer[visual]

\definevisualcounter
  [visualitem]
  [scratchmarks]
  [
    counter=\getvalue{v_strc_itemgroups_counter},
    width=1.5bp,
    height=1.2ExHeight,
    distance=3bp,
  ]

\definesymbol[visual][\usevisualcounter{visualitem}]
\stopbuffer

\startbuffer[page]
\definevisualcounter
  [visualpage]
  [mayanumbers]
  [
    counter=userpage,
    maxwidth=\textwidth,
  ]

\stopbuffer

\startbuffer[pagenumber]
\setupfootertexts[\usevisualcounter{visualpage}]
\stopbuffer


\getbuffer[visual,page,pagenumber]

\startbuffer[brightred]
\definecolor[bright-red] [h=DE1B1B]
\definecolor[dull-black] [h=2B2B2B]
\definecolor[dull-yellow][h=E9E581]

\definepalet
   [brightred]
   [past=dull-black,
    active=bright-red,
    future=dull-yellow]
\stopbuffer

\def\showcounter[#1]%
    {\startmiddlealigned
      \startcombination[#1]
        {\usevisualcounter[n=3, last=12]{dummy}}{3 out of 12}
        {\usevisualcounter[n=4, last=12]{dummy}}{4 out of 12}
        {\usevisualcounter[n=5, last=12]{dummy}}{5 out of 12}
        {\usevisualcounter[n=6, last=12]{dummy}}{6 out of 12}
      \stopcombination
    \stopmiddlealigned}

\starttext


\title {The \command{visualcounter} module \\ Aditya Mahajan \\ currentdate}

\starttyping
\usemodule[visualcounter]
\stoptyping

\startitemize[visual, broad, packed, columns, two]
  \item Find \TEX\ documents to be too boring?
  \item Voilà, the \command{visualcounter} module!
  \item Make your presentations stand out.
  \item Turn any counter into a picture.
\stopitemize

The above effect was achieved by first defining a \command{visualitem} counter
and a symbol \command{visual} that uses that counter:

\typebuffer[visual]

and then using the symbol \command{visual} in an itemization:

\starttyping
\startitemize[visual, ...]
    \item ...
    \item ...
    ...
\stopitemize
\stoptyping

Notice the counter used for page numbering? That was achieved by first defining
a \command{visualpage} counter:

\typebuffer[page]

and setting it as the footer text:

\typebuffer[pagenumber]

The above examples show the basic usage of the modules. The module provides two
commands: \type{\definevisualcounter} to define a visual counter

\starttyping
\definevisualcounter
    [...]     % [[\comment{name of the counter}]]
    [...]     % [[\comment{optional name of the parent counter} ]]
    [
     ...=..., % [[\comment{key-value settings} ]]
    ]
\stoptyping

and \type{\usevisualcounter} to use an already defined counter

\starttyping
\usevisualcounter
  [...=...] % [[\comment{key-value settings}]]
  {...}     % [[\comment{name of the counter}]]
\stoptyping

\subject {So, how do I use this?}

Visual counters are defined and used in two ways:
\startitemize[visual, broad, joinedup]
  \item Using a low level interface that explicitly sets the current values of
    the counter, last count, the \METAPOST\ graphic that draws the counter, and
    the color palette. 
  \item Using higher-level interfaces built on top of the low-level interface
    that allows you to specify a \emph{structure counter} like those used for
    page numbering, itemizations, descriptions, etc.
\stopitemize

\subsubject {The low-level interface}

To begin with, lets not worry about how to define \METAPOST\ graphics that draw
the counter. The module provides a predefined set of \command{visualcounter}s,
and, for now, we'll just use those: the \command{scratchmarks}  counter. Details
on defining new counters is in \in{Section}[sec:new-counters].

\startbuffer[scratchmarks]
\usevisualcounter[n=6, last=12]{scratchmarks}
\stopbuffer

Suppose that I want to show that I am on page 6 out of 12 pages:

\startmiddlealigned
  \getbuffer[scratchmarks]
\stopmiddlealigned

which uses a predefined counter \command{scratchmarks} and was typed as follows:

\typebuffer[scratchmarks]

The counter may be made smaller

\startbuffer[smaller]
\setupvisualcounter[scratchmarks]
                   [width=1pt, height=8pt, distance=2pt]
\stopbuffer

\startmiddlealigned
  \getbuffer[smaller,scratchmarks]
\stopmiddlealigned

or may use a different color palette

\startbuffer[palette]
\setupvisualcounter[scratchmarks][palette=brightred]
\stopbuffer

\startmiddlealigned
  \getbuffer[brightred,palette,scratchmarks]
\stopmiddlealigned

These settings are changed using \command{\setupvisualcounter}. In particular, to
get a small counter, use: 

\typebuffer[smaller]

and to change the color palette, use:

\typebuffer[palette]

where the \type{brightred} palette was defined as

\typebuffer[brightred]

\subsubject {The high-level interface}

\section {The \command{scratchmarks} counter}

The \command{scratchmarks} counter is inspired by the \command{fuzzycount}
counter that is part of the ConTeXt's metapost library \command{txt}. The output
looks as follows:

\definevisualcounter
  [dummy]
  [scratchmarks]

\showcounter[4]

The \command{scratchmarks} counter has the following tunable parameters:
\startitemize[packed, joinedup]
  \item \options{width} (default \options{3bp}): the width of each stroke
  \item \options{height} (default \options{3ExHeight}): 
    the length of the marker (and strictly speaking, not
    the height; the real height is $\options{height}×\sin(\options{angle})$).
  \item \options{distance} (default \options{0.5EmWidth}):
    the distance between two successive markers. The
    distance is measured from the middle of one marker to the middle of the
    other (that is, it does not take the width of the stroke into account).
  \item \options{angle} (default \options{75}): the angle of the forward
    markers. The angle of the backward marker is \command{-angle}.
    Only angles between \options{-90} and \options{90} give proper output. 
\stopitemize

For example, the output with \options{width=1.5bp, angle=45} is:

{\setupvisualcounter[dummy][width=1.5bp, angle=45]\showcounter[4]}

An angle less than \options{0} changes the direction of the stroke. 
For example the output with \options{width=1.5bp, angle=-45} is:

{\setupvisualcounter[dummy][width=1.5bp, angle=-45]\showcounter[4]}


\section {The \command{mayanumbers} counter}

The \command{mayanumbers} counter is inspired by the Mayan numbering system that I saw in the
documentary \quotation{Breaking the Maya code}. It counter does not strictly
follow the Mayan numbering system. The Mayan numbering system is written
vertically; the output of this counter is horizontal which makes it more useful
for displaying page numbers in presentations.

\definevisualcounter
  [dummy]
  [mayanumbers]

\showcounter[2*2]

The shape of the small and the large markers is as follows:

\startmiddlealigned
  \startcombination[2]
    {\usevisualcounter[n=1,last=5,width=3EmWidth, height=3ExHeight]{dummy}}
    {The shape of the small marker}
    {\usevisualcounter[n=5,last=5,width=3EmWidth, height=3ExHeight]{dummy}}
    {The shape of the large marker}
  \stopcombination
\stopmiddlealigned

The \command{scratchmarks} counter has the following tunable parameters:
\startitemize[packed, joinedup]
  \item \options{width} (default value \options{1EmWidth}): The width of the
    small marker. 
  \item \options{height} (default value \options{1ExHeight}): The height of the
    markers.
  \item \options{distance} (default value \options{0.25EmWidth}): The distance
    between two small markers. The distance between each group of four small
    counters is \command{2*distance}.
\stopitemize

For example, to get an output that is half the default size, use 
the options \options{width=0.5EmWidth, height=0.5ExHeight,
distance=0.125EmWidth}.

{\setupvisualcounter[dummy][width=0.5EmWidth, height=0.5ExHeight, distance=0.125EmWidth]\showcounter[4]}

\section {The \command{countdown} counter}


\definevisualcounter
  [dummy]
  [countdown]

\showcounter[4]


\section {The \command{markers} counter}


\definevisualcounter
  [dummy]
  [markers]

\showcounter[2*2]

\section {The \command{pulseline} counter}

\definevisualcounter
  [dummy]
  [pulseline]

\showcounter[2*2]


\section[sec:new-counters]{How to define new counters}

\stoptext
